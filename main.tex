%%%%%%%%%%%%%%%%%%%%%%%%%%%%%%%%%%%%%%%%%
%
% Lucerne University of Applied Sciences and Arts
% Version 0.4 - 11-03-2025
%
% HSLU-I: Official Thesis Template
%
% This template complies with the official guidelines for bachelor thesis provided
% on complesis by the department HSLU-I (computer science).
% It supports the languages english and german. The language can be set in
% the section: Document Information
%
% Original guidelines can be found on complesis:
% https://complesis.hslu.ch/
%
% Original author:
%   Ramón Christen HSLU-I
%
%%%%%%%%%%%%%%%%%%%%%%%%%%%%%%%%%%%%%%%%%


%----------------------------------------------------------------------------------------
% PACKAGES AND DOCUMENT CONFIGURATIONS
%----------------------------------------------------------------------------------------
\documentclass[12pt, a4paper, oneside]{book}    % document style definition
\usepackage{hslu}                               % apply HSLU style
\usepackage{comment}                            % having comment sections \begin{comment} \end{comment}
\usepackage[utf8]{inputenc}                     % charactere interpretation
\usepackage{amsmath}                            % math package
\usepackage{amsfonts}                           % font package for math symbols
\usepackage{amssymb}                            % symbols package - definition of math symbols
\usepackage{listings}                           % package for code representation
\usepackage{graphicx}                           % for inclusion of image
\usepackage{subfig}                             % to arrange figures next to each other
\usepackage{float}                              % text style surrounding images
\usepackage[acronym]{glossaries}                % package for glossary
\usepackage{tikz}                               % used to place logos on title page
% \usepackage{gensymb}                          % for special characters such as °
\usepackage[a-1a]{pdfx}                         % Forces PDF/A-1a compliance for long-term archiving


\usepackage{multirow}
\usepackage{siunitx}
\usepackage{tabularx}
\usepackage{tikzscale}


\hypersetup{hidelinks}                          % hide red border in hyperlinks
\setcounter{tocdepth}{1}                        % hide subsections from TOC
\makenoidxglossaries
% \DeclareAcronym{hslu}{short=HSLU, long=Lucerne University of Applied Sciences and Arts}
\newacronym{hslu}{HSLU}{Lucerne University of Applied Sciences and Arts}
\newacronym{nn}{NN}{Neural Network}
                                % include acronyms.txt file
\newglossaryentry{cognitive computing}
{
    name={Cognitive computing},
    description={A set of theories and techniques to let computers to mimic the mechanisms of the human brain. It provides the basis for the practical application of cognition and learning theories to computer systems with the use of soft computing methods.}
}

\newglossaryentry{perceptual computing}
{
    name={Perceptual computing},
    description={A set of theories and techniques allowing computers to compute and reason with perceptions and imprecise data.}
}

\newglossaryentry{fuzzy logic}
{
    name={Fuzzy logic},
    description={An extension of classical binary logic, where the truth value of propositions can not only be completely true or false, but also partially true and false to varying degree.}
}

\newglossaryentry{cww}
{
    name={Computing with words and perceptions},
    description={A process allowing to perform computations on words, phrases, and prepositions drawn from a natural language, which describe perceptions of people towards different aspects of the context they are surrounded by. This is based on the fuzzy logic toolbox and allows to represent and perform operations on the meaning of words.}
}

\newglossaryentry{convolutional neural network}
{
    name={Convolutional neural network},
    description={A class of neural networks commonly used for image analysis that is relying on convolution operations to extract features from data.}
}
                                % include glossary.txt file
\graphicspath{{figs/}}          % set path of graphics folder


%----------------------------------------------------------------------------------------
% PDF/A DOCUMENT COMPLIANCE
%----------------------------------------------------------------------------------------
\pdfcatalog{
  /StructTreeRoot <<                            % Define the structure tree root for document tagging
    /Type /StructTreeRoot                       % Specify that this is a structure tree root
    /K []                                       % Placeholder for structure elements (empty for now)
  >>
  /MarkInfo << /Marked true >>                  % Ensure the document is marked as tagged for accessibility
}


\begin{document}
%----------------------------------------------------------------------------------------
% DOCUMENT INFORMATION
%----------------------------------------------------------------------------------------
\thesisLanguage{english}                        % set thesis language [english, german]
\author{Author Name}                            % author name
\city{Lucerne (Switzerland)}                    % author's place of origin
\title{Thesis Title}                            % thesis title
\subtitle{\large subtitle}                      % thesis subtitle

\date{2024}                                     % the year when the thesis was written (used in titlepage)
\defensedate{October 27th, 2024}                % the date of the private defense
\defencelocation{Lucerne}                       % location of defence
\extexpert{Expert Name}                         % name of external expert
\indpartner{Company Name}                       % name of industry partner

% jury, supervisor and dean are only relevant if acceptance sheet is enabled with the next line
% \addAcceptsheet
\jury{                                          % members of the jury
    \begin{itemize}
        \item Prof. Dr. Name Surname from Lucerne University of Applied Sciences and Arts, Switzerland (President of the Jury);
        \item Prof. Dr. Name Surname from Lucerne University of Applied Sciences and Arts, Switzerland (Thesis Supervisor);
        \item Prof. Dr. Name Surname from Lucerne University of Applied Sciences and Arts, Switzerland (External Expert).
    \end{itemize}
}

\supervisor{Prof. Dr. Name Surname}             % name of supervisor
\dean{Prof. Dr. Name Surname}                   % name of faculty dean

\acknowledgments{Thanks to my family, relatives and firends for all the support given to finish this thesis.}



%----------------------------------------------------------------------------------------
% BEGIN DOCUMENT AND CREATE TITLEPAGE
%----------------------------------------------------------------------------------------
\maketitle



%----------------------------------------------------------------------------------------
% PREAMBLE
%----------------------------------------------------------------------------------------
\begin{abstractstyle}{\hsummary}
    The content of your thesis in brief.
\end{abstractstyle}

\tableofcontents
\listoffigures
\listoftables
% print list of acronyms and glossary
\printnoidxglossaries



%----------------------------------------------------------------------------------------
% MAIN CONTENT
%----------------------------------------------------------------------------------------
\mainmatter

% examplary content: write or compose the main document here
\chapter{Main Content}
This is a template of \gls{hslu} and then. This section usually comprises different chapters and subchapters.

\section{First Section}
Followed by a brief introduction, the section may comprise several subsections explaining various concepts and refering to external results \cite{christen_exogenous_2020}.

\subsection{First Subsection}
Content may also refer to special expressions such as that has to be explained in a separate section. Here we also have \gls{fuzzy logic} enough space and \gls{cww} to discuss the concept of \gls{nn} and so on.

\begin{comment}
    This is a sample comment.
\end{comment}

\bibliographystyle{ieeetr}
\footnotesize\bibliography{references}



%----------------------------------------------------------------------------------------
% APPENDIX
%----------------------------------------------------------------------------------------
\appendix

\glsaddallunused                                % add all unused items to glossary

\end{document}
